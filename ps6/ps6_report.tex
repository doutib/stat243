\documentclass{llncs}\usepackage[]{graphicx}\usepackage[]{color}
%% maxwidth is the original width if it is less than linewidth
%% otherwise use linewidth (to make sure the graphics do not exceed the margin)
\makeatletter
\def\maxwidth{ %
  \ifdim\Gin@nat@width>\linewidth
    \linewidth
  \else
    \Gin@nat@width
  \fi
}
\makeatother

\definecolor{fgcolor}{rgb}{0.345, 0.345, 0.345}
\newcommand{\hlnum}[1]{\textcolor[rgb]{0.686,0.059,0.569}{#1}}%
\newcommand{\hlstr}[1]{\textcolor[rgb]{0.192,0.494,0.8}{#1}}%
\newcommand{\hlcom}[1]{\textcolor[rgb]{0.678,0.584,0.686}{\textit{#1}}}%
\newcommand{\hlopt}[1]{\textcolor[rgb]{0,0,0}{#1}}%
\newcommand{\hlstd}[1]{\textcolor[rgb]{0.345,0.345,0.345}{#1}}%
\newcommand{\hlkwa}[1]{\textcolor[rgb]{0.161,0.373,0.58}{\textbf{#1}}}%
\newcommand{\hlkwb}[1]{\textcolor[rgb]{0.69,0.353,0.396}{#1}}%
\newcommand{\hlkwc}[1]{\textcolor[rgb]{0.333,0.667,0.333}{#1}}%
\newcommand{\hlkwd}[1]{\textcolor[rgb]{0.737,0.353,0.396}{\textbf{#1}}}%

\usepackage{framed}
\makeatletter
\newenvironment{kframe}{%
 \def\at@end@of@kframe{}%
 \ifinner\ifhmode%
  \def\at@end@of@kframe{\end{minipage}}%
  \begin{minipage}{\columnwidth}%
 \fi\fi%
 \def\FrameCommand##1{\hskip\@totalleftmargin \hskip-\fboxsep
 \colorbox{shadecolor}{##1}\hskip-\fboxsep
     % There is no \\@totalrightmargin, so:
     \hskip-\linewidth \hskip-\@totalleftmargin \hskip\columnwidth}%
 \MakeFramed {\advance\hsize-\width
   \@totalleftmargin\z@ \linewidth\hsize
   \@setminipage}}%
 {\par\unskip\endMakeFramed%
 \at@end@of@kframe}
\makeatother

\definecolor{shadecolor}{rgb}{.97, .97, .97}
\definecolor{messagecolor}{rgb}{0, 0, 0}
\definecolor{warningcolor}{rgb}{1, 0, 1}
\definecolor{errorcolor}{rgb}{1, 0, 0}
\newenvironment{knitrout}{}{} % an empty environment to be redefined in TeX

\usepackage{alltt}
\usepackage{listings}
\usepackage{moreverb}
\usepackage{inconsolata}
\pagestyle{plain}


\IfFileExists{upquote.sty}{\usepackage{upquote}}{}
\begin{document}

\title{Problem Set 6}
\author{Thibault Doutre, Student ID 26980469}
\institute{STAT243 : Statistical Computing\\
University of California, Berkeley}
\date{\today}
\maketitle
\bigbreak
\noindent
I worked on my own.
%%%%%%%%%%%%%%%%%%%%%%%%%%%%%%%%%%%%%%%%%%%%%%%%%%%%%%%%%%%
\section{Airline Database}
%%%%%%%%%%%%%%%%%%%%%%%%%%%%%%%%%%%%%%%%%%%%%%%%%%%%%%%%%%%%
I create a script which I can execute via EC2. The script builds a database "airline.db" and store every file into a single table, using RSQLite. In order to do it, I first download the files in a directory named "data" and create the database. In this database, I store a table called "airline" in wich I will append the data for every year.
\begin{knitrout}
\definecolor{shadecolor}{rgb}{0.969, 0.969, 0.969}\color{fgcolor}\begin{kframe}
\begin{alltt}
\hlcom{#Download}
\hlstd{url} \hlkwb{=} \hlstr{'http://www.stat.berkeley.edu/share/paciorek/1987-2008.csvs.tgz'}
\hlkwd{download.file}\hlstd{(url,} \hlstr{".file"}\hlstd{)}

\hlcom{# Untar}
\hlkwd{untar}\hlstd{(}\hlstr{".file"}\hlstd{,} \hlkwc{compressed} \hlstd{=} \hlstr{'bzip2'}\hlstd{,} \hlkwc{exdir} \hlstd{=} \hlstr{"./data/"}\hlstd{)}

\hlcom{# Create Database}
\hlkwd{library}\hlstd{(RSQLite)}
\hlstd{drv} \hlkwb{<-} \hlkwd{dbDriver}\hlstd{(}\hlstr{"SQLite"}\hlstd{)}
\hlstd{db} \hlkwb{<-} \hlkwd{dbConnect}\hlstd{(drv,} \hlkwc{dbname} \hlstd{=} \hlstr{"airline.db"}\hlstd{)}

\hlcom{# Create airline Table}
\hlkwd{dbSendQuery}\hlstd{(}\hlkwc{conn} \hlstd{= db,}
            \hlstr{[1325 chars quoted with '"']}
\hlstd{)}
\end{alltt}
\end{kframe}
\end{knitrout}
Then, for every year:
\begin{itemize}
\item Open a connection to the file with bzcat 
\item Get the data and store it into a variable "line"
\item Create a temporary table from line
\item Append this table to "airline" using INSERT
\item Remove the temporary table
\item Close the connection
\end{itemize}
\begin{knitrout}
\definecolor{shadecolor}{rgb}{0.969, 0.969, 0.969}\color{fgcolor}\begin{kframe}
\begin{alltt}
\hlcom{# Append years into airline table}
\hlkwa{for} \hlstd{(i} \hlkwa{in} \hlnum{1987}\hlopt{:}\hlnum{2008}\hlstd{)\{}
  \hlstd{con}\hlkwb{=}\hlkwd{pipe}\hlstd{(}\hlkwd{paste}\hlstd{(}\hlstr{"bzcat "}\hlstd{,i,}\hlstr{".csv.bz2"}\hlstd{,}\hlkwc{sep}\hlstd{=}\hlstr{""}\hlstd{),} \hlkwc{open} \hlstd{=} \hlstr{'r'}\hlstd{)}
  \hlstd{lines} \hlkwb{=} \hlkwd{read.csv}\hlstd{(con,} \hlkwc{header} \hlstd{=} \hlnum{TRUE}\hlstd{)}
  \hlkwd{dbWriteTable}\hlstd{(db,} \hlkwd{paste}\hlstd{(}\hlstr{"y"}\hlstd{,i,}\hlkwc{sep}\hlstd{=}\hlstr{""}\hlstd{),lines)}
  \hlkwd{dbSendQuery}\hlstd{(db,}\hlkwd{paste}\hlstd{(}\hlstr{"INSERT INTO airline SELECT * FROM y"}\hlstd{,i,}\hlkwc{sep}\hlstd{=}\hlstr{""}\hlstd{))}
  \hlkwd{dbRemoveTable}\hlstd{(db,}\hlkwd{paste}\hlstd{(}\hlstr{"y"}\hlstd{,i,}\hlkwc{sep}\hlstd{=}\hlstr{""}\hlstd{))}
  \hlkwd{close}\hlstd{(con)}
\hlstd{\}}
\end{alltt}
\end{kframe}
\end{knitrout}
Then, I print the size of the database:
\begin{knitrout}
\definecolor{shadecolor}{rgb}{0.969, 0.969, 0.969}\color{fgcolor}\begin{kframe}
\begin{alltt}
\hlcom{# Size in Gb}
\hlkwd{file.size}\hlstd{(}\hlstr{"./airline.db"}\hlstd{)}\hlopt{/}\hlnum{2}\hlopt{^}\hlnum{30}

\hlcom{# 9.273604}
\end{alltt}
\end{kframe}
\end{knitrout}
We can see that the database is 9 Gb big, it is less than the original CSV of 12 Gb but significantly bigger than the bzipped file of 1.7 Gb.

\end{document}


